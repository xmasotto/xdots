\documentclass{article}
% Almost always {article}. 
% [options]: http://www.nada.kth.se/~carsten/latex/class.html

\usepackage{enumerate}
\usepackage{amsmath, amsthm, amssymb}

\usepackage {parskip}
\usepackage[margin=2.5cm]{geometry}
% http://tug.ctan.org/tex-archive/macros/latex/contrib/geometry/geometry.pdf

\begin{document}

\section{Part 1}

\subsection{Lists}

\noindent Here's my list using itemize:

\begin{itemize}
\item Item 1.
\item Item 2.
\item Item 3.
\end{itemize}

\noindent Here's my list using enumerate:

\begin{enumerate}
\item Item 1.
\item Item 2.
\item Item 3.
\end{enumerate}

\noindent Custom Labels

\begin{enumerate}[(a)]
\item 
  Hello
\item 
  World
\item 
  People
\end{enumerate}

\noindent Here's my list using description:

\begin{description}
\item[Heading 1.] Item 1.
\item[Heading 2.] Item 2.
\item[Heading 3.] Item 3.
\end{description}

\noindent Here's my list with nested itemize:

\begin{itemize}
\item Item 1.
\begin{itemize}
\item List 2, Item 1
\item List 2, Item 2
\end{itemize}
\item Item 2.
\item Item 3.
\end{itemize}


\subsection{Tables}
% \begin{tabular}[alignment (t or b)]{columns (l, c, r, or |)}
% rows
% \end{tabular}
% use \multicolumn to merge rows

\begin{tabular}[t]{|l|ccccc|c|}
\multicolumn{7}{c}{USAMTS Scores Round 1}\\\hline
Name&\#1&\#2&\#3&\#4&\#5&Total\\\hline
John Doe&5&5&3&2&1&16\\
Jane Doe&5&5&5&4&5&24\\
Richard Feynman&5&5&5&5&5&25\\\hline
\end{tabular}

\subsection{Boxes}

I can create basic boxes for text \makebox[2in]{like this}. Notice
that there's a 2in wide space with `like this' in the middle of it.

If I want to put a box around the text, I can use a frame box. The
result looks \framebox[2in]{like this}.

I can also justify the text to the right within a box
\makebox[1.5in][r]{like so} or \framebox[2.5in][l]{like so}.

We can also use quick versions of these. We can just \mbox{do this}
or \fbox{this} to create a quick box that's exactly the size of what we put in it.

\parbox[b]{2in}{I like using parbox to create funny little boxes of
text all over my page. This one has its bottom edge aligned with the
current line}
CURRENT LINE.
\parbox[t]{2in}{The top of my text is aligned with that current
line.}

\parbox{1.5in}{I'm just centered on the current line.} CURRENT LINE
\parbox{2.5in}{You probably got a few Overfull or Underfull warnings
when you typeset this. Sometimes narrow boxes will do that; if you're
happy with the output, don't sweat it.}

\subsection{Centering}

The helix hath spoken.

\begin{center}
Fourscore and seven years ago our fathers brought forth on this
continent a new nation, conceived in liberty and dedicated to the
proposition that all men are created equal.
\end{center}

\begin{flushleft}
Now we are engaged in a great civil war, testing whether that nation
or any nation so conceived and so dedicated can long endure. We are
met on a great battlefield of that war. We have come to dedicate a
portion of it as a final resting place for those who died here that
the nation might live. This we may, in all propriety do. But in a
larger sense, we cannot dedicate, we cannot consecrate, we cannot
hallow this ground. The brave men, living and dead who struggled here
have hallowed it far above our poor power to add or detract. The
world will little note nor long remember what we say here, but it can
never forget what they did here.
\end{flushleft}

\begin{flushright}
It is rather for us the living, we here be dedicated to the great
task remaining before us--that from these honored dead we take
increased devotion to that cause for which they here gave the last
full measure of devotion--that we here highly resolve that these dead
shall not have died in vain, that this nation shall have a new birth
of freedom, and that government of the people, by the people, for the
people shall not perish from the earth.
\end{flushright}

\section{Part 2}

\subsection{Size Matters}

When I was born, I was {\small small}. Actually, {\scriptsize I was
very small}. When I got older, I thought some day {\Large I would be
large}, {\Huge maybe even gigantic}. But instead, I'm not even
normalsize. {\small I'm still small.}


\subsection {Style over Substance}

When I was born, I was {\small small}. Actually, {\scriptsize I was
very small}. When I got older, I thought some day {\Large I would be
large}, {\Huge maybe even gigantic}. But instead, I'm not even
normalsize. {\small I'm still small.}

\subsubsection{Bold, Italics, Underline}

When something is \emph{really}, \textbf{really} important, you can
\underline{underline it}, \emph{italicize it}, \textbf{bold it}. If
you \underline{\textbf{\emph{must do all three}}}, then you can nest
them.

\subsubsection{Font Families}

You may want to write things \textsf{in a sans-serif font}, or
\texttt{in a typewriter font}, or \textsl{in a slanted font} (which
is \emph{slightly different} than italics). Sometimes it pays
\textsc{to write things in small capitals}. You can next go to
\textbf{bold and then \textsl{bold and slanted} and then back to just
bold} again.

\subsection{Spacing}

$x+y$

$x+\,y$

$x+\:y$

$x+\;y$

$x+\quad y$

$x+\qquad y$

$x+\!y$

I\hfill like \hfill space.
\vfill
In every direction.

I \hspace{2in} like \hspace{1in} space.

\vspace{5in}

In every direction.

\end{document}